% Technical report: ICARUS-DIARC integration
% Authors: Matthias Scheutz and Paul Schermerhorn

\documentclass{article}
\usepackage{verbatim}
\usepackage{graphicx}
\setlength{\textwidth}{6.0in}
\setlength{\oddsidemargin}{0.00in}
\setlength{\evensidemargin}{0.00in}
\setlength{\textheight}{9.00in}
\setlength{\topmargin}{0.00in}
\addtolength{\topmargin}{-\headheight}
\addtolength{\topmargin}{-\headsep}

\special{papersize=8.5in,11in}
\begin{document}
\title{IU TECHNICAL REPORT\\ICARUS-DIARC Action Integration\\Version 5}
\author{Matthias Scheutz and Paul Schermerhorn}

\maketitle

\section*{Introduction}

This technical report describes the integration of the cognitive
ICARUS architecture within the robotic DIARC architecture.  The idea
underlying the employed integration is that ICARUS figures as a {\em
  more complex component} in DIARC, which (1) uses explicit goal and
skill representations, (2) follows and pursues goals sequentially, (3)
focuses attention on percepts and internal states, (4) uses
problem-solving to cope with novel situations, and (5) learns new
hierarchical skills.  The integration is achieved via implementing a
special ``wrapper component'' -- the {\tt ICARUSActionInterpeter}
server -- which extends the standard DIARC action interpreter (in
conjunction with an extended ActionManager called {\tt
  ICARUSActionManager}) and provides and a special purpose interface
to ICARUS as described below.

DIARC can interact with ICARUS in four ways: (1) through ICARUS'
perceptual buffer by placing percepts into the buffer, (2) through a
novel goal placement mechanism where DIARC can place new goals
expressed in the CTL$^{*}$ formalism directly into ICARUS' goal
buffer, (3) via action requests from ICARUS called ``primitive
intentions'' (PI) that will generate goals in DIARC and will be
executed along side all other actions in DIARC, and (4) through an
introspective goal reflection mechanism (via special beliefs) that
allows ICARUS to check which of its primitive intentions are currently
being carried out and which were successful or failed.\footnote{Note
  that this is different from Version 1, where primitive actions in
  ICARUS were directly mapped onto actions scripts in DIARC via the
  special ICARUS action interpreter, and where action script failures
  could not be communicated back to ICARUS.}

The perceptual integration mechanism implements both implicit and
explicit attentional mechanisms that, on the one hand, allow ICARUS to
``focus its attention'' on particular percepts (e.g., ones that are of
relevance to its current task) by explicitly requesting them from
DIARC, while DIARC can ``re-direct'' ICARUS' attention implicitly by
placing unrequested percepts into the buffer that might be of
importance from its perspective (e.g., percepts relating to obstacle
avoidance when a new obstacle appeared in the front of the robot).
Moreover, the novel goal-passing mechanism allows DIARC to deposit
explicit goals into ICARUS' goal buffer on every cycle and thus
implements another, higher-level attention mechanism.  That way other
components in DIARC (e.g., the NLP component) can generate explicit
goal representations that ICARUS itself might not be able to generate
(e.g., goals generated from natural language commands).  Finally, the
revised action processing mechanism via {\em primitive intentions}
that now maps ICARUS action primitives directly onto DIARC goals
allows for a cycle-based issuing of DIARC goals that ICARUS wants
DIARC to pursue.  The associated reflection mechanism provides
information about the execution status of primitive intentions in the
form of beliefs that can be used by ICARUS to avoid reissuing ongoing
intentions (different from Version 1 where ICARUS caused DIARC actions
in the service of then ``primitive actions'' to be repeated).

Note that DIARC still manages all its other goals (through its
standard goal manager) and carries out all its other actions in
parallel to ICARUS.  It does, however, have a special {\em permanent
  goal} when the ICARUS server is instantiated, that of running
ICARUS.  Currently, both ICARUS and DIARC can have goals that are not
accessible within the other system, but DIARC goals not issued by
ICARUS could be made available to ICARUS through the same goal
feedback mechanism as used for primitive intentions.


\section*{ICARUS-DIARC Interface}

The LISP and JAVA interfaces between ICARUS and DIARC are defined in
the code repository in the files

\vspace{2mm}
\centerline{com/action/icarus/ICARUSadeclient.lisp}
\vspace{2mm}

\noindent on the LISP side and in

\vspace{2mm}
\centerline{com/action/IcarusActionInterpreter.java}
\vspace{2mm}

\noindent on the JAVA side.  The interface is based on a simple
message passing protocol that allows DIARC first to initialize ICARUS
based on start-up parameters (such as the ICARUS knowledge file that
needs to be loaded) and then to pass control to ICARUS which can run
its cycle-based updates in its own LISP sub-process polling and pushing
data from and to its associated {\tt ICARUSActionInterpeter} (i.e.,
the {\tt ICARUSActionInterpeter} takes care of invoking Lisp, loading
ICARUS, setting up the communication socket, etc.).  The functions
{\tt initialize-world} and {\tt update-world} are defined, but are not
currently used.

Percepts are given to ICARUS via the {\tt preattend} mechanism.  The 
types of percepts to which the ICARUS client wants to attend can be 
specified (see below), and the percepts are returned to {\tt 
preattend} as ordered lists of objects in the format:

\begin{verbatim}
    <object type> <object token> <slot> <value> <slot> <value> ...
\end{verbatim}

Figure~\ref{integration} shows ICARUS as running in the special
purpose DIARC {\tt ICARUSActionInterpeter} component, where primitive
intentions issued by ICARUS are instantiated as DIARC goals that cause
the executing of associated action scripts that interpreted in the
standard way for DIARC scripts.

\begin{figure}[t]
  \includegraphics[width=6in]{icarus-diarc2}
  \caption{Schematic of the ICARUS-DIARC integration via the special
    purpose {\tt ICARUSActionInterpeter} component in DIARC.}
  \label{integration}
\end{figure}


\subsection*{ICARUS Percepts}

A special ``attend'' DIARC primitive has been defined to allow ICARUS 
to specify which percepts are of interest:

\begin{verbatim}
  (ICARUS-intention (attend percept-type-1 percept-type-2...))
\end{verbatim}

\noindent This will focus attention on those types of percepts, which 
will then be subsequently sent to ICARUS through its {\tt preattend} 
call.  Several percepts are defined in the ICARUS 
interface:\footnote{Note that, for the sake of backward compatibility, 
many of the percepts below will include x-y coordinates as well, and 
many of the primitive intentions defined later will still accept x-y 
coordinates as parameters.  However, these are deprecated and could be 
removed in future versions of the interface.}

%  (obstacle ?name direction ?dir distance ?dist)
%     ; an obstacle in the given direction
%     ; ("left", "right", "infront") at the given distance

\begin{verbatim}
  (robot ?name orientation ?h velocity ?vel location ?l)
     ; the name, orientation, and velocity of the robot and its
     ; location, which is the name of the closest perceivable object

  (landmark ?name type ?t heading ?dir distance ?dist) 
     ; a perceived landmark of type t in relative direction to the 
     ; robot's heading in degrees at the distance in meters -- this
     ; will be a substitute for any type of landmark in the task as used
     ; in the director's map (e.g., a doorway, wall, chair, table, etc.),
     ; NOTE: landmarks are not movable or not pickupable, different from
     ; boxes and blocks

  (box ?name color ?value open? ?o filled? ?f heading ?dir distance ?dist)
     ; a perceived box with a color, a boolean indicating whether it
     ; is open or not, a boolean indicating whether it is empty or not
     ; (only if it is open), a direction relative to the robot's
     ; heading in degrees, and the distance in meters

  (block ?name color ?color inside ?box) 
     ; a perceived block with a color inside a particular box

\end{verbatim}

So, if ICARUS needs to get information about boxes, then it would
first attend to boxes---``(attend box)''---and then process any box
percepts returned in response.  The attend mechanism is persistent, so 
once a percept type has been attended to, DIARC will continue to send 
percepts of that type to ICARUS until that attend is cancelled by 
another attend.  Subsequent attends replace previous attend lists, so, 
for example, to attend also to blocks, it would be necessary to send 
``(attend box block)'' to DIARC.

\subsection*{ICARUS Goals}

DIARC can also explicitly place new goals in the goal buffer in
ICARUS.  This is intended to allow for future goals expressed in LTL
to be sent to ICARUS from other components of the system (e.g., the
NLP components).  For example, the natural language command ``go to
the break room and then report your location'' would be translated
into the LTL goal ``$\diamond$(at(breakroom) \&
`$\diamond$report(location,self))''.  This goal could then be placed
into the goal buffer in an appropriate LISP representation.

Note that while such goal representations could index skills directly
(e.g., if a skill was associated with the goal description in LTL), it
will in general be non-trivial to determine what skills to select to
achieve an LTL goal.  This, however, is a problem that needs to be
resolved within ICARUS.  DIARC here is only a mediator between an
external producer of a goal representation and ICARUS which has to
achieve the goal through the execution of skills.

\subsection*{ICARUS Intentions}

Actions to be performed by DIARC (i.e., outside the ICARUS
architecture) are requested by virtue of ``primitive intentions'' sent
by ICARUS to the {\tt ICARUSActionInterpeter} via the {\tt
  ICARUS-intention} function as:

\begin{verbatim}
    (ICARUS-intention (intention-type-name arg1 arg2 ...)
                      (intention-type-name arg1 arg2 ...))
\end{verbatim}

\noindent Note that the type names of primitive intentions are not
bound directly to Lisp functions, but are mapped onto DIARC goals
that, in turn, trigger associated action scripts that will be executed
(in parallel) outside of ICARUS.  The intention primitives (i.e.,
DIARC goals) available to all instances of the action manager are
defined in the file {\sc com/action/db/action.xml}.  For the purposes
of the {\tt ICARUS-intention} function call, the most important
components of the specification of a primitive intention are its name
and possible arguments.  Several primitive intentions have been
defined to allow ICARUS to re-create the different types of skills
(and strategies) exhibited by humans in the Search\&Rescue Task:

\begin{verbatim}
  (look-for ?type)
    ; looks around 360 degrees in current location without moving
    ; placing percepts of ?type into the perceptual buffer (NOTE: this
    ; is not equivalent to pre-attend) success: if 360 turn succeeds
    ; (even if no object is perceived), failure otherwise

  (move-to ?object)
    ; moves from current location to the location of the perceivable
    ; object (box, doorway, or other landmark); success if robot reaches
    ; the location of the object, failure otherwise (including if
    ; ?object does not refer to a currently perceivable object)

  (move)
    ; move forward until an object is detected in in front, while at
    ; the same time keeping away from objects on the left and right;
    ; success when robot is stopped by an obstacle in front (useful for
    ; moving across a room or down a hallway)

  (move-through ?doorway)
    ; moves through a perceivable doorway and stops right after it has
    ; passed the doorway

  (turn ?object)
    ; turns in place to face ?object, if known success if robot
    ; completes turn, failure otherwise

  (turn ?distance)
    ; turns in place ?distance degrees; success if robot completes
    ; turn, failure otherwise ("turn -90" == "turn right", "turn 180" ==
    ; "turn around", and "turn 90" == "turn left")

  (open ?box)
    ; opens a box; success if open, failure otherwise

  (close ?box)
    ; closes a box; success if closed, failure otherwise
             
  (get-from ?box ?block)
    ; picks up the block from the box; success if the robot holds the
    ; block, failure otherwise NOTE: robot can only hold one block at a
    ; time

  (put-into ?box ?block)
    ; puts the block in the container; success if block is in box,
    ; failure otherwise
 
  (pick-up ?box)
    ; picks up the box; success if robot holds the box, failure
    ; otherwise; NOTE: the robot can only hold one box at a time

  (put-down ?box)
    ; puts down the box in robot's current location; success if the box
    ; is down, failure otherwise

  (report ?object ?slot1 ?slot2 ...)
    ; reports for a given percept a list of slot values for the given
    ; slot names

\end{verbatim}

Additional actions related to NL interactions (e.g., with the human
director) can be added at a later point.

\subsection*{ICARUS Beliefs about Intentions}

Any action in DIARC can be in one of three states at any given time:
{\tt success} (i.e., successful completion of the action), {\tt
  failure} (i.e., action could not complete for whatever reason --
depending on the action there might be failure conditions that can be
reflected on), or {\tt ongoing} (i.e., the action is still ongoing and
has not completed yet).  Whenever a primitive intention is scheduled,
DIARC creates a new intention name and using this name it informs
ICARUS on subsequent cycles about the status of the intention as long
as the intention is ongoing (the last update is performed when the
intention succeeds or fails).  The updates about the status of
intentions are placed directly into ICARUS' belief buffer on each
cycle in the form:

\begin{verbatim}
  (primitive-intention (<intention-name> <status>)
                       (<intention-name> <status>)
                       ...)
\end{verbatim}

More specifically, when a PI has been issued for the first time, DIARC
assigns it a new token identifier (via a gensym call), which it
subsequently uses to refer to that very intention.  So, if one cycle
$t$ ICARUS issued the PI

\begin{verbatim}
  (look-for "box" color "green")
\end{verbatim}

then DIARC assigns this a unique identifier ``look-for17'' which it 
will subsequently report to ICARUS' belief memory:

\begin{verbatim}
  (primitive-intention ("look-for17" "ongoing"))
\end{verbatim}

This way ICARUS can determine that it previously had that intention
and branch on its state (e.g., ``ongoing'') accordingly.  Note that
this mechanisms can be used to prevent ICARUS from issuing the same
intention over and over again (e.g., the intention to move forward if
the robot is already moving forward).

\subsection*{Reference Implementation}

Included below (and as {\tt com/action/icarus/test3.lisp} in the 
repository) is a Lisp implementation of an agent that solves the 
``report green boxes'' and ``gather yellow blocks'' goals of the task 
using the ICARUS-DIARC interface.  It first goes all the way through 
the environment to the cardboard box, gets that, then moves back 
through the environment, visiting the blue boxes and removing the 
yellow blocks from them and placing them into the cardboard box.  It 
concurrently monitors for green boxes, reporting them as they are 
found.  After identifying all the green boxes and removing all the 
yellow blocks from blue boxes, it returns to the starting position.

Each step of the code represents a spoken language instruction (e.g., 
"go through the nearest doorway"), without requiring  knowledge ahead 
of time of any names of objects, instead getting them from the 
environment as they're encountered.  Hence, the solution could 
represent a sequence of instructions from the director, based on the 
map.

\begin{verbatim}

;;; Overwrites ICARUS's run function so we can test the interface
(defvar action-status* nil)
(defvar reported-boxes nil)
(defvar action nil)
(defvar actstate nil)
(defvar target nil)
(defvar oldtarget nil)
(defvar mybox nil)
(defvar myblock nil)
(defvar checkval nil)
(defvar tdist -91)
(ICARUS-intention (list (list "attend" "box" "block" "landmark")))
(defun run (runcycles)
  (preattend)
  ;;; go through the door
  (loop for perc in percepts do
        (if (equal (getslot perc 'type) 'doorway)
          (setq action (car (ICARUS-intention (list (list "move-through" (cadr perc))))))))
  (monitor action 'checkaction)
  ;;; go down the hallway to the steps
  (loop for perc in percepts do
        (if (equal (getslot perc 'type) 'steps)
          (setq action (car (ICARUS-intention (list (list "move-to" (cadr perc))))))))
  (monitor action 'checkaction)
  ;;; turn right
  (setq action (car (ICARUS-intention (list (list "turn" -90)))))
  (monitor action 'checkaction)
  ;;; go through the leftmost doorway
  (loop for perc in percepts do
        (if (and (equal (getslot perc 'type) 'doorway)
                 (> (getslot perc 'heading) tdist))
          (progn (setq target (cadr perc))
                 (setq tdist (getslot perc 'heading)))))
  (setq action (car (ICARUS-intention (list (list "move-through" target)))))
  (monitor action 'checkaction)
  ;;; move to the cart
  (loop for perc in percepts do
        (if (equal (getslot perc 'type) 'cart)
          (progn (setq target (cadr perc)))))
  (setq action (car (ICARUS-intention (list (list "move-to" target)))))
  (monitor action 'checkaction)
  ;;; turn to the doorway
  (loop for perc in percepts do
        (if (equal (getslot perc 'type) 'doorway)
          (setq target (cadr perc))))
  (setq action (car (ICARUS-intention (list (list "turn" target)))))
  (monitor action 'checkaction)
  ;;; go through the doorway
  (setq action (car (ICARUS-intention (list (list "move-through" target)))))
  (monitor action 'checkaction)
  ;;; turn right
  (setq action (car (ICARUS-intention (list (list "turn" -90)))))
  (monitor action 'checkaction)
  ;;; go down the hall...
  (setq action (car (ICARUS-intention (list (list "move")))))
  ;;; ...until you see the box (landmark of type box--a little confusing)
  (monitor action (lambda (actname)
                    (loop for perc in percepts do
                          (if (equal (getslot perc 'type) 'box)
                            (progn (setq target (cadr perc))
                                   (return t))))))
  ;;; then go past the box (to the box)...
  (setq action (car (ICARUS-intention (list (list "move-to" target)))))
  (monitor action 'checkaction)
  ;;; ...and on to the end of the hall
  (setq action (car (ICARUS-intention (list (list "move")))))
  (print (list 'action action-status*))
  (monitor action 'checkaction)
  ;;; when you get to the end of the hall, turn left
  (setq action (car (ICARUS-intention (list (list "turn" 90)))))
  (monitor action 'checkaction)
  ;;; go to the brown box
  (loop for perc in percepts do
        (if (and (equal (car perc) 'box)
                 (equal (getslot perc 'color) 'brown))
          (progn (setq target (cadr perc))
                 (setq action (car (ICARUS-intention (list (list "move-to" target))))))))
  (monitor action 'checkaction)
  ;;; face the box
  (setq action (car (ICARUS-intention (list (list "turn" target)))))
  (monitor action 'checkaction)
  ;;; and pick it up
  (setq mybox target)
  (setq action (car (ICARUS-intention (list (list "pick-up" target)))))
  (monitor action 'checkaction)
  ;;; open the box
  (setq action (car (ICARUS-intention (list (list "open" target)))))
  (monitor action 'checkaction)
  ;;; turn around
  (setq action (car (ICARUS-intention (list (list "turn" 180)))))
  (monitor action 'checkaction)
  ;;; move to the end of the hall
  (setq action (car (ICARUS-intention (list (list "move")))))
  (monitor action 'checkaction)
  ;;; when you get to the end of the hall, turn right
  (setq action (car (ICARUS-intention (list (list "turn" -90)))))
  (monitor action 'checkaction)
  ;;; move down the hall...
  (setq action (car (ICARUS-intention (list (list "move")))))
  ;;; ...until you see a blue box
  (monitor action (lambda (actname)
                    (loop for perc in percepts do
                          (if (and (equal (car perc) 'box)
                                   (equal (getslot perc 'color) 'blue))
                            (progn (setq target (cadr perc))
                                   (return t))))))
  ;;; then go to it
  (setq action (car (ICARUS-intention (list (list "move-to" target)))))
  (monitor action 'checkaction)
  ;;; get the yellow block
  (setq action (get-block target))
  (monitor action 'checkaction)
  ;;; turn left
  (setq action (car (ICARUS-intention (list (list "turn" 90)))))
  (monitor action 'checkaction)
  ;;; move down the hall...
  (setq action (car (ICARUS-intention (list (list "move")))))
  ;;; ...until you see a blue box
  (monitor action (lambda (actname)
                    (loop for perc in percepts do
                          (if (and (equal (car perc) 'box)
                                   (equal (getslot perc 'color) 'blue))
                            (progn (setq target (cadr perc))
                                   (return t))))))
  ;;; then go to it
  (setq tdist 500)
  (loop for perc in percepts do
        (if (and (equal (car perc) 'box)
                 (equal (getslot perc 'color) 'blue)
                 (< (getslot perc 'distance) tdist))
          (progn (setq target (cadr perc))
                 (setq tdist (getslot perc 'distance)))))
  (setq action (car (ICARUS-intention (list (list "move-to" target)))))
  (monitor action 'checkaction)
  ;;; get the yellow block
  (setq action (get-block target))
  (monitor action 'checkaction)
  ;;; turn left
  (setq action (car (ICARUS-intention (list (list "turn" 90)))))
  (monitor action 'checkaction)
  ;;; go through the door
  (loop for perc in percepts do
        (if (equal (getslot perc 'type) 'doorway)
          (setq action (car (ICARUS-intention (list (list "move-through" (cadr perc))))))))
  (monitor action 'checkaction)
  ;;; turn right
  (setq action (car (ICARUS-intention (list (list "turn" -90)))))
  (monitor action 'checkaction)
  ;;; go to the blue box
  (loop for perc in percepts do
        (if (and (equal (car perc) 'box)
                 (equal (getslot perc 'color) 'blue))
          (progn (setq target (cadr perc))
                 (setq action (car (ICARUS-intention (list (list "move-to" target))))))))
  (monitor action 'checkaction)
  ;;; get the yellow block
  (setq action (get-block target))
  (monitor action 'checkaction)
  ;;; turn left
  (setq action (car (ICARUS-intention (list (list "turn" 90)))))
  (monitor action 'checkaction)
  ;;; go through the closest door
  (setq tdist 500)
  (loop for perc in percepts do
        (if (and (equal (getslot perc 'type) 'doorway)
                 (< (getslot perc 'distance) tdist))
          (progn (setq target (cadr perc))
                 (setq tdist (getslot perc 'distance)))))
  (setq action (car (ICARUS-intention (list (list "move-through" target)))))
  (monitor action 'checkaction)
  ;;; turn left...
  (setq action (car (ICARUS-intention (list (list "turn" 180)))))
  ;;; ...until you see a blue box
  (monitor action (lambda (actname)
                    (loop for perc in percepts do
                          (if (and (equal (car perc) 'box)
                                   (equal (getslot perc 'color) 'blue))
                            (progn (setq target (cadr perc))
                                   (return t))))))
  ;;; get the yellow block
  (setq action (get-block nil))
  (monitor action 'checkaction)
  ;;; turn left
  (setq action (car (ICARUS-intention (list (list "turn" 90)))))
  (monitor action 'checkaction)
  ;;; go through the door
  (loop for perc in percepts do
        (if (equal (getslot perc 'type) 'doorway)
          (progn (setq target (cadr perc))
                 (setq action (car (ICARUS-intention (list (list "move-through" target))))))))
  (monitor action 'checkaction)
  ;;; turn left
  (setq action (car (ICARUS-intention (list (list "turn" 90)))))
  (monitor action 'checkaction)
  ;;; go through the next-closest door on the left
  (setq tdist 500)
  (setq oldtarget target)
  (loop for perc in percepts do
        (if (and (equal (getslot perc 'type) 'doorway)
                 (< (getslot perc 'distance) tdist)
                 (not (equal (cadr perc) oldtarget)))
          (progn (setq target (cadr perc))
                 (setq tdist (getslot perc 'distance)))))
  (setq action (car (ICARUS-intention (list (list "move-through" target)))))
  (monitor action 'checkaction)
  ;;; go to the blue box
  (loop for perc in percepts do
        (if (and (equal (car perc) 'box)
                 (equal (getslot perc 'color) 'blue))
          (progn (setq target (cadr perc))
                 (setq action (car (ICARUS-intention (list (list "move-to" target))))))))
  (monitor action 'checkaction)
  ;;; get the yellow block
  (setq action (get-block target))
  (monitor action 'checkaction)
  ;;; turn right
  (setq action (car (ICARUS-intention (list (list "turn" -90)))))
  (monitor action 'checkaction)
  ;;; go through the closest door
  (setq tdist 500)
  (loop for perc in percepts do
        (if (and (equal (getslot perc 'type) 'doorway)
                 (< (getslot perc 'distance) tdist))
          (progn (setq target (cadr perc))
                 (setq tdist (getslot perc 'distance)))))
  (setq action (car (ICARUS-intention (list (list "move-through" target)))))
  (monitor action 'checkaction)
  ;;; go to the blue box on the left
  (loop for perc in percepts do
        (if (and (equal (car perc) 'box)
                 (equal (getslot perc 'color) 'blue)
                 (> (getslot perc 'heading) 0))
          (progn (setq target (cadr perc))
                 (setq action (car (ICARUS-intention (list (list "move-to" target))))))))
  (monitor action 'checkaction)
  ;;; get the yellow block
  (setq action (get-block target))
  (monitor action 'checkaction)
  ;;; turn around
  (setq action (car (ICARUS-intention (list (list "turn" 180)))))
  (monitor action 'checkaction)
  ;;; go to the blue box
  (loop for perc in percepts do
        (if (and (equal (car perc) 'box)
                 (equal (getslot perc 'color) 'blue))
          (progn (setq target (cadr perc))
                 (setq action (car (ICARUS-intention (list (list "move-to" target))))))))
  (monitor action 'checkaction)
  ;;; get the yellow block
  (setq action (get-block target))
  (monitor action 'checkaction)
  ;;; turn left...
  (setq action (car (ICARUS-intention (list (list "turn" 180)))))
  ;;; ...until you see a doorway
  (monitor action (lambda (actname)
                    (loop for perc in percepts do
                          (if (equal (getslot perc 'type) 'doorway)
                            (progn (setq target (cadr perc))
                                   (return t))))))
  ;;; go through the doorway
  (setq tdist 500)
  (loop for perc in percepts do
        (if (and (equal (getslot perc 'type) 'doorway)
                 (< (getslot perc 'distance) tdist))
          (progn (setq target (cadr perc))
                 (setq tdist (getslot perc 'distance)))))
  (setq action (car (ICARUS-intention (list (list "move-through" target)))))
  (monitor action 'checkaction)
  ;;; and go through the next doorway
  (setq tdist 500)
  (loop for perc in percepts do
        (if (and (equal (getslot perc 'type) 'doorway)
                 (< (getslot perc 'distance) tdist))
          (progn (setq target (cadr perc))
                 (setq tdist (getslot perc 'distance)))))
  (setq action (car (ICARUS-intention (list (list "move-through" target)))))
  (monitor action 'checkaction)
  ;;; turn right
  (setq action (car (ICARUS-intention (list (list "turn" -90)))))
  (monitor action 'checkaction)
  ;;; get the yellow block
  (setq action (get-block nil))
  (monitor action 'checkaction)
  ;;; turn right
  (setq action (car (ICARUS-intention (list (list "turn" -90)))))
  (monitor action 'checkaction)
  ;;; go through the doorway
  (setq tdist 500)
  (loop for perc in percepts do
        (if (and (equal (getslot perc 'type) 'doorway)
                 (< (getslot perc 'distance) tdist))
          (progn (setq target (cadr perc))
                 (setq tdist (getslot perc 'distance)))))
  (setq action (car (ICARUS-intention (list (list "move-through" target)))))
  (monitor action 'checkaction)
  ;;; turn left
  (setq action (car (ICARUS-intention (list (list "turn" 90)))))
  (monitor action 'checkaction)
  ;;; go to the end of the hall
  (setq action (car (ICARUS-intention (list (list "move")))))
  (monitor action 'checkaction)
  ;;; turn left
  (setq action (car (ICARUS-intention (list (list "turn" 90)))))
  (monitor action 'checkaction)
  ;;; go through the doorway
  (setq tdist 500)
  (loop for perc in percepts do
        (if (and (equal (getslot perc 'type) 'doorway)
                 (< (getslot perc 'distance) tdist))
          (progn (setq target (cadr perc))
                 (setq tdist (getslot perc 'distance)))))
  (setq action (car (ICARUS-intention (list (list "move-through" target)))))
  (monitor action 'checkaction)
  ;;; and move to the end of the hall
  (setq action (car (ICARUS-intention (list (list "move")))))
  (monitor action 'checkaction)
  (quit)
  ) ;;; end run

;;; check for green boxes in percept list, report if found, add to list
(defun monitor (actname checkfunc)
  (loop until actstate do
    (ICARUS-intention (list (list "sleep" 50)))
    (preattend)
    (print percepts)
    (print action-status*)
    (loop for perc in percepts do
          (if (and (equal (car perc) 'box) 
                   (equal (getslot (cddr perc) 'color) 'green))
            (report (cadr perc))))
    (setq actstate (funcall checkfunc actname))
    (force-output *standard-output*))
  (setq actstate nil))

(defun checkaction (actname)
  (loop for act in (cdr action-status*) do
        ;;;(print (list "checking" act "for" actname))
        (if (equal (car act) action)
          (if (equal (cadr act) 'success)
            (return-from checkaction t)
            (return-from checkaction nil))))
  nil)

(defun getslot (perc sname)
  ;;; scan the list for slot of sname, return its value or nil if unfound
  (cond ((equal (car perc) sname)
         (cadr perc))
        ((null (cddr perc))
         nil)
        (t (getslot (cddr perc) sname))))

(defun report (name)
  (loop for box in reported-boxes do
        (if (equal name box)
          (return-from report nil)))
  (setq reported-boxes (cons name reported-boxes))
  (ICARUS-intention (list (list 'report name "color" "heading" "distance"))))

(defun get-block (target)
  (if (null target)
    ;;; get the closest blue box
    (progn (setq tdist 500)
           (loop for perc in percepts do
                 (if (and (equal (car perc) 'box)
                          (equal (getslot perc 'color) 'blue)
                          (< (getslot perc 'distance) tdist))
                   (progn (setq tdist (getslot perc 'distance))
                          (setq target (cadr perc)))))))
  ;;; face the box
  (setq action (car (ICARUS-intention (list (list "turn" target)))))
  (monitor action 'checkaction)
  ;;; open the box
  (setq action (car (ICARUS-intention (list (list "open" target)))))
  (monitor action 'checkaction)
  ;;; grab the yellow block
  (loop for perc in percepts do
        (if (and (equal (car perc) 'block)
                 (equal (getslot perc 'color) 'yellow))
          (progn (setq myblock (cadr perc))
                 (setq action (car (ICARUS-intention (list (list "get-from" target myblock))))))))
  (monitor action 'checkaction)
  ;;; put it into the cardboard box
  (setq action (car (ICARUS-intention (list (list "put-into" mybox myblock)))))
  (monitor action 'checkaction)
  (car (ICARUS-intention (list (list "close" target)))))
\end{verbatim}

\end{document}



boxes

blocks

robot <name> location <>


==> (container <name> color <value> location <location>) ; allocentric location of container

(container blue-box-1 color blue location location-17)

==> when that percepts is created, we add an assocation between location-17 and the location in the map


  -- only one container can be picked up at any time
  -- 






LET'S assume that we have landmarks annotated before hand (this is
only to deal with the perceptual problems); e.g., we might have doors
indicated in the map, and hallways, but only as types...

SO: DIARC CAN SEE ``DOORS'' from the MAP



  (pick-up cardboard-box1)

  (open ?blue-box1) ; afterwards you can see the content

  (inspect-content ?blue-box1)


  


  (report ?property ?object) - reports the ?property of the ?object

   (ICARUS-intention '("startmove" "icarus"))
   (ICARUS-intention '("pick-up" "icarus"))


    <!-- Stop -->
    <name>stop</name>
    <desc>?mover stops</desc>
    <dbkind>action</dbkind>
    <role>
      <rolename>mover</rolename>
      <roletype>actor</roletype>
    </role>
    <locks>motionLock</locks>
    <postcond>not(moving(mover))</postcond>
    <postcond>not(turning(mover))</postcond>
    <taskname>Stop</taskname>
  </type>
\end{verbatim}

In this case, the name of the action is {\tt stop}, and it has one 
role (i.e., argument), {\tt mover}.  This should be {\tt icarus} in 
this case, as the name of the ICARUS agent is ``icarus'' by fiat.  The 
remaining components of the primitive definition are not of interest 
to {\tt ICARUS-intention}.  Note that some roles are optional (e.g., the 
{\tt extent} and {\tt units} roles of {\tt startmove} do not need to 
be specified, in which case the robot will just start moving and keep 
moving until told to stop, or until obstacle avoidance is engaged.





 (move-to ?location)
[13:31:56] … (move-to 'location1)
[13:32:20] … (look-for "box" 'color "blue")
[13:33:54] … (report 'location blue-box1)
[13:34:15] … (the location of blue box 1 is 19,19)
[13:34:29] … (report 'color blue-box1)
[13:34:36] … (the color of blue-box1 is blue)
[13:34:49] … (report 'size icarus)
[13:34:57] … (sorry, icarus does not have a size)
[13:35:12] … (sorry, robot percepts don't have size slots)


--> look for all blue boxes (implicitly "in all rooms")?

